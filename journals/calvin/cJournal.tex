\documentclass{article}
\author{Calvin Golas}
\title{Journal Entries}

\begin{document}

{\Large Calvin Golas}\\

\section{\Large \itshape  Entry 1, 6/13/19}
    I looked into Tensorflow today,
\newline
Feels good to be doing what I love and getting pay.
\newline
Meeting is tomorrow we'll see
\newline
How this project shapes up to be.

\section{\Large \itshape  Entry 2, 6/14/19}
    Meet up with Daniel and the gang
\newline
Topology entices me and R is an interesting lang.
\newline
I have a lot of work ahead of me in learning the math and tech,
\newline
But in the end we'll have a pelican recognizer up to spec.
\newline


\section{\Large \itshape  Entry 3, 6/15/19}
    Met with the previous project lad,
\newline
The syntax of tensor code sure looks bad.
\newline
I'm hopeful though seeing all of the possibilities we have ahead,
\newline
I'll try to figure the basics out Saturday before I go to bed.

\section{\Large \itshape  Entry 4, 6/15/19}
    Functors and tensorflow all were learned,
\newline
Extreme learning curve I must say.
\newline
Through the knowledge I have earned,
\newline
I can code with the 'flow with a thought and prayer.

\section{\Large \itshape  Entry 5, 6/17/19}
    Tensorflow and topology were dived into,
\newline
Everything starting to click together.
\newline
My respect for toruses and line theory greatly grew.
\newline
Gaining interest in strange math again altogether.

\section{\Large \itshape  Entry 6, 6/18/19}
    Four hour commute aside,
\newline
Kathy's warnings of abstraction are now handy.
\newline
Keras or tensorflow, github tutorials I have eyed,
\newline
Looking at examples that catch the eye on medium like candy.

\section{\Large \itshape  Entry 7, 6/19/19}
    A book I am reading,
\newline
faster and faster about linear algebra.
\newline
My professor's advice I am heeding.
\newline
Stoked to continue, on to local maxima!

\section{\Large \itshape  Entry 8, 6/20/19}
    Just finished the joint meeting with Jonas and co,
\newline
I like that they're using pytorch, it was an interesting tool.
\newline
I want to focus on object classification, here we go.
\newline
Going to try implementing yolo in tensorflow, that'd be cool.

\section{\Large \itshape  Entry 9, 6/22/19}
    Tis the age of ssh upon us,
\newline
And here I am, many bits and bytes of imagery.
\newline
Of what you may ask, I won't make you guess.
\newline
Many many pelicans, no more no less.

\section{\Large \itshape  Entry 10, 6/24/19}
    Tried looking at Yolo implementation,
\newline
It didn't work as darknet is strange.
\newline
On to Detectron we go without trepadation,
\newline
From tensorflow to pytorch I change.

\section{\Large \itshape Entry 11, 6/25/19}
    Object detection really is a treat,
\newline
And any documentation none can beat. (NoteL: Especially in how confusing it is!)
\newline
Single shot detection and tensorflow object detection API is sweet
\newline
Rarely had I learned so much out of my depth, it truly is neat.

\section{\Large \itshape Entry 12, 6/26/19}
    After hours of vision research,
\newline
I've found retraining to be quite key.
\newline
Less processing intensive I found by search,
\newline
And to boot, it'll work! (probably)

\section{\Large \itshape Entry 13, 6/27/19}
    Learned what seperates bounding boxes and masks.
\newline
They're both good, but for our tasks,
\newline
It seems we should go with bounding boxes.
\newline
Marking up every edge in a picture? Now that's obnoxious.

\section{\Large \itshape Entry 14, 6/28/19}
    Listened to a lot of podcast episodes,
\newline
That lead down a lot of roads.
\newline
Found out about supervisely.
\newline
That dataset will make our R-CNN work, probably.

\section{\Large \itshape Entry 15, 6/29/19}
    Learned about best practices I have,
\newline
Google's object detection finally installed.
\newline
Transfer learning will cause work to halve.
\newline
In the documentation I briefly stalled.

\section{\Large \itshape Entry 16, 7/1/19}
    Object detection I have working.
\newline
Webcam up and now detecting.
\newline
Code up on github I have been collecting.
\newline
This will work I think for 'icans lurking.

\section{{\Large \itshape Entry 17, 7/2/19}
    Group coding we have been doing,
\newline
People closing in on visions.
\newline
Bananas we have all been chewing,
\newline
Images bounding boxes will soon have their divisions.

\section{{\Large \itshape Entry 18, 7/3/19}
    Labelled some pelicans I did,
\newline
Over a hundred in fact.
\newline
If I am to perish others can benefit
\newline
To train a bounding box ssd act.

\section{{\Large \itshape Entry 19, 7/5/19}
    Setup the model,
\newline
Created the tf records correctly.
\newline
The whole process does my mind boggle.
\newline
Hopefully when this training is done pelicans will be seen directly.

\section{{\Large \itshape Entry 20, 7/8/19}
    Tested the model I have done,
\newline
And I must say not bad.
\newline
Some pelicans were classified like something out of tron.
\newline
Now I need to retrain with images from pelicam I had.

\section{{\Large \itshape Entry 21, 7/9/19}
    Model I have presented,
\newline
Relearn I must have it do.
\newline
Transfer to server files defected,
\newline
Now on to the training I go.

\section{{\Large \itshape Entry 21, 7/9/19}
    Environment on the server I have done,
\newline
Cross training can happen, of doubt there is none.
\newline
I read from my book, I wait on each step.
\newline
Soon our project will fill with pelican identifying pep.

\end{document}