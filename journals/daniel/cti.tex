\chapter{Notes on Type Hierarchy Investigation}

I read  ``A Category Theoretical Investigation of the Type Hierarchy for
Heterogeneous Sensor Integration Notes'' \cite{purvine2016category}. It's a lot
of good work done, providing transformations for data types into concrete
categories.\footnote{A concrete category is one for which there exists a
faithful functor to the category of sets.} The categorification of common data
types is done to streamline the definition of sheaves, functors whose
destination categories are concrete categories. So, readers will be allowed to
substitute their own sensor data as long as they conform to basic data types,
such as a boolean or a random variable.

The underlying assumption of the paper is that the researcher wants to know the
state of something, and is taking measurements to that purpose. The state of
that something can be expressed as a set of variables taking some value. The
measurements are provided by some sensors. So, the goal is to convert the raw
data from sensors into a set of values; because a sheaf is being used, the data
from each sensor is at some point combined with data from other sensors. The
values that the variables take must be expressible as elements of some set.
The workflow streamlined by this paper is as follows:
\begin{enumerate}
	\item Identify data sources as sensors. Sensor is a general concept,
	just something that produces data, like a newspaper article or a
	thermometer. Let $S$ be the set of sensors.
	\item \label{pairStep} Identify the variables that each sensor can
	inform about. A sensor may convey information about more than one
	variable. Let $V$ be the set of variables.
	\item  For each pair $(s_i \in S, v_j \in V)$ identified in step
	\ref{pairStep}, convert the raw data into an intermediate form. The
	authors call this an \emph{analytic}. The intermediate form is a bit of
	an art. It has to be some form that can then be converted to one of the
	following data types 
	\begin{itemize} 
		\item number
		\item real interval
		\item ordinal
		\item partial order
		\item stochastic process
		\item random variable
		\item probability distribution
		\item measure (mathematical)
		\item binary relation
		\item n-ary relation
		\item boolean
	\end{itemize}
	\item Convert the analytic into one of the data types.
	\item From here on, Purvine et al. provide a conversion from these data
	types into a category, and furthermore a functor to the category of
	sets.
\end{enumerate}

So as far a sheaves (a functor from a posetal category to a concrete category)
are concerned, all we need to do is specify a domain category, where each
sensor is a local section and the readings from all sensors form a global
section. 

Sheaves have been defined from a variety of domains, including
\begin{itemize} 
	\item  a topological space\cite{topsheafWolfram},
	\item a site\cite{siteSheaf}, and
	\item a posetal category\cite{robinson2017sheaf}.
\end{itemize}

When defining a sheaf on a topological space, the domain category is one where
objects are the open sets of the topology, and the morphisms are inclusions
subset into superset. As far as sensors are concerned, I'm guessing we can
begin with a simplicial complex, where 0-simplices are sensors and we include a
$n$-dimensional simplex where $n-1$ simplices agree. For example, if two
sensors produce analyitics that can be related to the same variable, there
exists an edge between them.

We apply the Alexandrov topology\footnote{I should caveat this by saying that
I'm not totally convinced I've interpreted the Alexandrov topology correctly
yet}, that is we designate as open sets all the upper sets of each simplex. So
a simplex does not exist in an open set without its star.

The open sets can then, as specified, be designated as objects in a category
with inclusion morphisms. The restrictions will therefore follow the opposite
direction of inclusion: we will restrict data from superset to subset.

Now consider two sensors that produce an analytic on the same variable: A and
B.  The simplex is A, B, and the edge AB. So the upperset of A, $u(A)$, will be
$\{A,AB\}$. The intersection of $u(A)$ and $u(B)$ is also included as an open
set and is just the edge $\{AB\}$ (which should be included anyway since
$u(AB)$ is $\{AB\}$). So we restrict down from the stalk of $u(A)$ onto the
stalk of $u(AB)$ since $u(AB) \hookrightarrow u(A)$. This is where the magic
happens. Defining a restriction from both $u(A)$ and $u(B)$ is composing both
sensors. The sheaf condition\cite{spivak2014category} then allows us to say
that the union of all open sets (i.e. $\{A,B,AB\}$) has a unique state that can
be mapped all the way down from the stalk of $\{A,B,AB,\}$ into the stalk of
$\{AB\}$.

It is my belief that while finding a good merging technique is an interesting
problem, co-homology will tell us something about the global states and it is
something I should investigate.
