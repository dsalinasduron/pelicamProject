\documentclass{article}
\author{Jake Wagoner}
\title{Journal Entries}

\begin{document}

{\Large Jake Wagoner}\\

\section{\Large \itshape  Entry 1, June 13, 2019}
	Today I did some preliminary research into Tensorflow and general machine learning.
There are a lot of things I need to learn still but it's important that I stay on task and learn the important things.
One question I have is about what language/system I should start diving into, but that information will likely be revealed at tomorrow's meeting.
I am somewhat concerned that this will all go over my head, but I think that's also just part of the process. The math is fairly complicated but there are libraries for things like this I'm sure.


\section{\Large \itshape Entry 2, June 14, 2019}
	Today we met with Jack and discussed the work he had previously done on the project.
It was really interesting to learn how he conducted some of the research. I have been doing some research into R, and it's not my favorite.
I think finding a python library that implements Keras or using Keras in python would be the best course of action moving forward.

\section{\Large \itshape Entry 3, June 17, 2019}
	Today I read into Functors and their applications in mathematics. I will need to do further research into how functors are used in Machine Learning.
I also spend a lot of time going through images in Zooniverse and identifying Pelicans. It helped me to have a further understanding of what kind of images we will be processing and the challenges that come with it.

\section{\Large \itshape Entry 4, June 18, 2019}
	Today we met to discuss how things will be moving forward. We were able to go further into depth on many subjects like functors. We also discussed our course of action moving forward.
This entailed a sort of division of labor in which I will be trying to look into dividing images up into smaller pieces to pass through the image processing software Jack already made.
This will allow us to isolate small parts of images with and without Pelicans, so we could potentially use this to count Pelicans.

\section{\Large \itshape Entry 5, June 19, 2019}
	I spent most of today trying to dig into the server files to find Jack's code so I could try and run it for myself.
The goal was to get in so I could work on splitting images up into smaller pieces to determine if they contained pelicans. I wasn't able to gain access to any of that, as far as I could find.
I also did a lot of digging into methods to split images into smaller pieces using cropping or chopping. Almost all the viable solutions and libraries I found were written in python. This isn't an issue since the server has python, the issue is communication between the R code and python code.
I did find that there is an R package for advanced image processing in R called Magick. This does have the ability to crop, so I will be doing some further reading into this package if we do end up working heavily with R.		

\end{document}
