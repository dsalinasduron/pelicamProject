\documentclass{article}
\usepackage{times}
\begin{document} 
\title{\LARGE{\textbf{dear diary}}}
\author{ali}
\maketitle
\newpage
\tableofcontents
\newpage

\section{6/14/2019}
for the first log lets go over what I did overall til now.
I looked at "Train your first neural network"
tried doing it on vscode but tensorflow didnt work and in my attempts to fix it by changing python version, 
I somehow managed to totally fuck it up and now I have bigger problems.
I started watching the google machine learning video. Im half way through so Im gonna talk more about that in the next commit ( tomorrow or sunday).
I also started looking at some latex tutorials. it seems to be too much work just to make a pdf file.
I don't like github as much as others but its a major tool so I better get used to it. every single time i go on it it feels like a new experience cuz i forget everything every time.
I feel a little scared and maybe overwhelmed but it be like that sometimes. its just the beginning.
(you will probably find alot of mistakes (grammatical or verbal) in my logs because grammarly does not work on vscode)

\section{6/17/2019}
today I completed the google ML crash course to the end of "First Steps with TF".
I already had an idea what they were talking about because of my statistical modeling class. So the material was not entirely new to me. 
It all went well up to the last section. I opened the exercise for TensorFlow and went through half of it before the code stopped working.
It just refused to acknowledge the previous work it had done itself and just gave me errors. Even when refreshed the page and did it all again.
But anyway. It was cool and now I’m really interested in doing the rest of the tutorial and see what it says.

\section{6/18/2019}
today started with me trying to find a document or a video on YouTube about functors that could actually understand. I found a 1hr video of a guy who was explaining what a functor is and what is its significance. But the level of complexity was too high for me. But then I found another video by the same guy on functors in programming which intrigued me. So now I want to go and watch the first video again and try to understand and then watch the “functors in programming”.
I also read a paper about a group of researchers that like us had cameras taking pictures of animals. But we want a machine learning algorithm that counts the number of animals in a picture but their machine learning algorithms only recognizes an animal in a picture. But still we could look at what they done and learn from it. Like I am now responsible to do some research on resnet 18 architecture and see if we can use it.
But there has to be more papers and research done that is close to what we are doing. We just have to search for it.

\end{document}