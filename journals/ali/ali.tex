\documentclass{article}
\usepackage{times}
\begin{document} 
\title{\LARGE{\textbf{dear diary}}}
\author{ali}
\maketitle
\newpage
\tableofcontents
\newpage

\section{6/14/2019}
for the first log lets go over what I did overall til now.
I looked at "Train your first neural network"
tried doing it on vscode but tensorflow didnt work and in my attempts to fix it by changing python version, 
I somehow managed to totally fuck it up and now I have bigger problems.
I started watching the google machine learning video. Im half way through so Im gonna talk more about that in the next commit ( tomorrow or sunday).
I also started looking at some latex tutorials. it seems to be too much work just to make a pdf file.
I don't like github as much as others but its a major tool so I better get used to it. every single time i go on it it feels like a new experience cuz i forget everything every time.
I feel a little scared and maybe overwhelmed but it be like that sometimes. its just the beginning.
(you will probably find alot of mistakes (grammatical or verbal) in my logs because grammarly does not work on vscode)

\section{6/17/2019}
today I completed the google ML crash course to the end of "First Steps with TF".
I already had an idea what they were talking about because of my statistical modeling class. So the material was not entirely new to me. 
It all went well up to the last section. I opened the exercise for TensorFlow and went through half of it before the code stopped working.
It just refused to acknowledge the previous work it had done itself and just gave me errors. Even when refreshed the page and did it all again.
But anyway. It was cool and now I’m really interested in doing the rest of the tutorial and see what it says.

\section{6/18/2019}
today started with me trying to find a document or a video on YouTube about functors that could actually understand. I found a 1hr video of a guy who was explaining what a functor is and what is its significance. But the level of complexity was too high for me. But then I found another video by the same guy on functors in programming which intrigued me. So now I want to go and watch the first video again and try to understand and then watch the “functors in programming”.
I also read a paper about a group of researchers that like us had cameras taking pictures of animals. But we want a machine learning algorithm that counts the number of animals in a picture but their machine learning algorithms only recognizes an animal in a picture. But still we could look at what they done and learn from it. Like I am now responsible to do some research on resnet 18 architecture and see if we can use it.
But there has to be more papers and research done that is close to what we are doing. We just have to search for it.

\section{6/19/19}
I did research on resnets especially resnet 18. It seems like an extra step that you can put on a CNN that makes it a little more efficient but does not make it anymore mathematically intensive. I think if we find out how exactly we can implement it and if we have the time to do so, we should. because it doesn't seem to have any drawbacks.

\section{6/20/19}
today we met with Jonas's team. I still don't understand what they want to do and what is their purpose. but they want to help us and I’m cool with that.

\section{6/21/19}
today I started working on the machine learning tutorial that Emily talked about. there is some inconsistency between Jonas was talking about and what’s on the website. the concepts are explained pretty well but the code is really messy and there is no straightforward walkthrough for the code.

\section{6/24/19}
I gave ML tutorial another code another shot and made some progress, but I think there is a very basic problem with the code that wants to access parts of pytorch that are not implemented. I’m not sure if the tutorial is actually up to date.

\section{6/25/19}
in the meeting we discussed the tutorial I went through and apparently Emily has the same issue as me. So I guess I’m not the only one. I will however start doing some more research more specifically on the topic of object detection. and make sure it can detect more than 1 object on a frame.

\section{6/26/19}
I found some interesting stuff called R-CNN, Fast R-CNN and Faster R-CNN. seems like it does exactly what we want but it is not as clear to me how it does so. this is some really intense stuff.

\section{6/27/19}
nothing much going on. just another meeting. but turns out Calvin is doing exactly what I wanted to do. I think he is working on R-CNNs too. I will jump on that ship and start looking for stuff I can code which are related to object detection.

\section{6/28/19}
as of right now I haven’t done much. I am just logging my hours and writing what was missing from my journal. But I will start looking at tutorials and papers and GitHub repositories to understand how a R-CNN architecture looks like and works.

\end{document}